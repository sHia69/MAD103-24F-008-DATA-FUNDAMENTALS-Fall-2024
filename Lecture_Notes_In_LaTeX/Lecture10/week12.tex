\documentclass{article}
\usepackage{amsmath}
\usepackage{fancyhdr}

% Header
\pagestyle{fancy}
\fancyhf{}
\fancyhead[L]{MAD 103 - Data Fundamentals}
\fancyhead[R]{Instructor: Sodiq Shofoluwe \thepage}

\begin{document}

\title{Table Joins in SQL}
\author{Hia Al Saleh}
\date{November 18th, 2024}
\maketitle
\tableofcontents
\newpage

\section{Introduction to Relational Databases}
Relational databases create relationships between two or more tables. This is useful when storing related information. For example:
\begin{itemize}
    \item We want to store information about a product and its supplier.
    \item A simple product table can include attributes like \texttt{prod\_name}, \texttt{qty}, and \texttt{supplier}.
    \item As data grows, storing all supplier contact details in the same table can become cumbersome and redundant.
\end{itemize}

\section{Creating Tables}
Below are examples of how to create tables for products and suppliers:
\subsection{Product Table}
\begin{verbatim}
CREATE TABLE product(
    id INTEGER NOT NULL AUTO_INCREMENT PRIMARY KEY,
    prod_name VARCHAR(30) NOT NULL,
    qty SMALLINT NOT NULL,
    supplier VARCHAR(10)
) ENGINE=INNODB;
\end{verbatim}

\subsection{Extended Product Table with Supplier Information}
\begin{verbatim}
CREATE TABLE product(
    id INTEGER NOT NULL AUTO_INCREMENT PRIMARY KEY,
    prod_name VARCHAR(30) NOT NULL,
    qty SMALLINT NOT NULL,
    supplier VARCHAR(10),
    contactName VARCHAR(50),
    contactPhone CHAR(10),
    contactEmail VARCHAR(30),
    contactPosition VARCHAR(15)
) ENGINE=INNODB;
\end{verbatim}

\subsection{Supplier Table}
\begin{verbatim}
CREATE TABLE supplier(
    id INTEGER NOT NULL AUTO_INCREMENT PRIMARY KEY,
    supplierName VARCHAR(30),
    contactName VARCHAR(50),
    contactPhone CHAR(10),
    contactEmail VARCHAR(30),
    contactPosition VARCHAR(15)
);
\end{verbatim}

\section{Inserting Data into Tables}
To insert data into these tables, the following SQL statements are used:

\subsection{Inserting Data into the Supplier Table}
\begin{verbatim}
INSERT INTO supplier
(supplierName, contactName, contactPhone, contactEmail, contactPosition)
VALUES
('Farm Chicken Supplier', 'John Doe', '555-6656', 'jdoe@email.com', 'Buyer'),
('Cow Farms', 'Mike Smith', '666-9656', 'msmith@email.com', 'Manager');
\end{verbatim}

\subsection{Inserting Data into the Product Table}
\begin{verbatim}
INSERT INTO product
(prod_name, qty, supplier)
VALUES
('Chicken', 2, 1),
('Turkey', 14, 1),
('Beef', 22, 2);
\end{verbatim}

\section{Retrieving Information}
\begin{itemize}
    \item Use the \texttt{SELECT} statement to retrieve information from tables.
    \item Information can be pulled from multiple tables using joins.
\end{itemize}

\section{Types of Joins}
Joins allow tables to be related row by row, satisfying specific conditions.

\subsection{INNER JOIN}
An \texttt{INNER JOIN} returns only the rows that satisfy the condition specified in the \texttt{ON} clause.

\begin{verbatim}
SELECT column1, column2, column3
FROM table_a INNER JOIN table_b
ON column_x = column_y;
\end{verbatim}

Example using the product and supplier tables:
\begin{verbatim}
SELECT *
FROM product INNER JOIN supplier
ON product.supplier = supplier.id;
\end{verbatim}

\subsection{OUTER JOIN}
An \texttt{OUTER JOIN} returns rows from at least one of the tables, even if there is no match in the other table.

\subsubsection{LEFT OUTER JOIN}
Includes all rows from the left table and the matched rows from the right table.

\begin{verbatim}
SELECT a.au_fname, a.au_lname, p.pub_name
FROM authors AS a LEFT OUTER JOIN publishers AS p
ON a.city = p.city;
\end{verbatim}

\subsubsection{RIGHT OUTER JOIN}
Includes all rows from the right table and the matched rows from the left table.

\begin{verbatim}
SELECT a.au_fname, a.au_lname, p.pub_name
FROM authors AS a RIGHT OUTER JOIN publishers AS p
ON a.city = p.city;
\end{verbatim}

\section{Querying with Conditions}
Use conditions to filter and join tables:
\begin{itemize}
    \item Use \texttt{WHERE} to specify filtering conditions.
    \item Use \texttt{ORDER BY} to sort the results.
\end{itemize}

Example:
\begin{verbatim}
SELECT products.productName, suppliers.companyName
FROM products INNER JOIN suppliers
ON products.SupplierID = suppliers.SupplierID
WHERE country = 'United States'
ORDER BY suppliers.companyName;
\end{verbatim}

\section{Group Data with Aggregations}
To group data and perform aggregations, the \texttt{GROUP BY} clause is useful.

Example:
\begin{verbatim}
SELECT s.companyName, COUNT(p.productName) AS numberOfProducts
FROM suppliers AS s INNER JOIN products AS p
ON s.SupplierID = p.SupplierID
GROUP BY s.companyName
ORDER BY numberOfProducts;
\end{verbatim}

\section{Summary}
\begin{itemize}
    \item Joins are a fundamental concept in SQL to combine data from multiple tables.
    \item \texttt{INNER JOIN} retrieves matching rows, while \texttt{OUTER JOIN} includes non-matching rows.
    \item Understanding joins is essential for efficient data querying and manipulation.
\end{itemize}

\end{document}