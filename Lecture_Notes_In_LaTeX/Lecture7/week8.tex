\documentclass{article}
\usepackage{amsmath}
\usepackage{fancyhdr}

% Header
\pagestyle{fancy}
\fancyhf{}
\fancyhead[L]{MAD 103 - Data Fundamentals}
\fancyhead[R]{Instructor: Sodiq Shofoluwe \thepage}

\begin{document}

\title{Functions and Date Operations in SQL}
\author{Hia Al Saleh}
\date{October 21st, 2024}
\maketitle
\tableofcontents
\newpage

\section{Date Functions}
\subsection{CURDATE() and CURRENT\_DATE}
\begin{itemize}
    \item \texttt{CURDATE()} and \texttt{CURRENT\_DATE} are synonyms for retrieving the current date.
\end{itemize}

\subsection{CURTIME() and CURRENT\_TIME}
\begin{itemize}
    \item Use \texttt{CURTIME()} or its synonym \texttt{CURRENT\_TIME} to retrieve the current time, formatted in a 24-hour clock.
\end{itemize}

\subsection{Timestamp and Datetime Extraction}
\begin{itemize}
    \item The \texttt{TIMESTAMP} function takes a datetime value and a time value to return a datetime.
    \item Use \texttt{EXTRACT()} to retrieve specific components such as:
    \begin{itemize}
        \item \texttt{YEAR}, \texttt{MONTH}, \texttt{DAY}
        \item \texttt{HOUR}, \texttt{MINUTE}, \texttt{SECOND}
    \end{itemize}
\end{itemize}

\subsection{Date Manipulation and Differences}
\begin{itemize}
    \item \texttt{DATEDIFF()} calculates the difference in days between two dates.
    \item Example:
    \begin{verbatim}
        SELECT username, info_updated, 
               DATEDIFF(CURRENT_DATE, info_updated) AS difference
        FROM test
        WHERE DATEDIFF(CURRENT_DATE, info_updated) > 200
        ORDER BY difference DESC;
    \end{verbatim}
    \item \texttt{DATE\_ADD()} allows you to add a specific interval to a date.
\end{itemize}

\newpage

\section{String Operations}
\subsection{Changing Case: UPPER() and LOWER()}
\begin{itemize}
    \item \texttt{UPPER()} converts strings to uppercase, and \texttt{LOWER()} converts them to lowercase.
    \item Example:
    \begin{verbatim}
        SELECT UPPER(name), LOWER(name)
        FROM members;
    \end{verbatim}
\end{itemize}

\subsection{Concatenating Strings}
\begin{itemize}
    \item \texttt{CONCAT()} combines multiple strings into one without adding spaces between them.
    \item Example:
    \begin{verbatim}
        SELECT CONCAT(first_name, ' ', last_name) AS Name
        FROM customer_data;
    \end{verbatim}
\end{itemize}

\subsection{Handling NULLs with COALESCE()}
\begin{itemize}
    \item \texttt{COALESCE()} helps handle \texttt{NULL} values in concatenation.
    \item Example:
    \begin{verbatim}
        SELECT CONCAT(name, ' ', COALESCE(suffix, ''))
        FROM members;
    \end{verbatim}
\end{itemize}

\newpage

\section{Arithmetic and Mathematical Operations}
\begin{itemize}
    \item Arithmetic operators such as \texttt{+}, \texttt{-}, \texttt{*}, and \texttt{/} are used for mathematical operations.
    \item Example:
    \begin{verbatim}
        SELECT grocery, vendor, qty, buy_price, 
               qty * buy_price AS totalSpent
        FROM groceries
        ORDER BY totalSpent DESC;
    \end{verbatim}
\end{itemize}

\subsection{Extracting Substrings}
\begin{itemize}
    \item Use \texttt{SUBSTRING()} to extract a part of a string.
    \item Example:
    \begin{verbatim}
        SELECT name, SUBSTRING(name FROM 1 FOR 1) AS firstLetter
        FROM animalNames
        ORDER BY firstLetter;
    \end{verbatim}
\end{itemize}

\subsection{Calculating String Length}
\begin{itemize}
    \item \texttt{CHARACTER\_LENGTH()} returns the number of characters in a string.
    \item Example:
    \begin{verbatim}
        SELECT name, CHARACTER_LENGTH(name) AS nameLength
        FROM animalNames
        ORDER BY nameLength DESC;
    \end{verbatim}
\end{itemize}

\subsection{Trimming Unwanted Characters}
\begin{itemize}
    \item \texttt{TRIM()} removes leading or trailing characters from a string.
    \item Example:
    \begin{verbatim}
        SELECT name, TRIM(LEADING 'S' FROM name)
        FROM animalNames
        WHERE SUBSTRING(name FROM 1 FOR 1) = 'S';
    \end{verbatim}
\end{itemize}

\end{document}
