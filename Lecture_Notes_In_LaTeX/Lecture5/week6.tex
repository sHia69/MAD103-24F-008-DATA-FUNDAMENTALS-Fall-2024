\documentclass{article}
\usepackage{amsmath}
\usepackage{fancyhdr}

% Header
\pagestyle{fancy}
\fancyhf{}
\fancyhead[L]{MAD 202 - Database Management}
\fancyhead[R]{Instructor: Sodiq Shofoluwe \thepage}

\begin{document}

\title{Data Manipulation Language (DML) in SQL}
\author{Hia Al Saleh}
\date{October 7th, 2024}
\maketitle
\tableofcontents
\newpage

\section{Introduction to DML}
Data Manipulation Language (DML) is a subset of SQL commands used to manage data in database tables. The primary DML commands are:
\begin{itemize}
    \item \texttt{INSERT} - Adds new records to a table.
    \item \texttt{UPDATE} - Modifies existing records.
    \item \texttt{DELETE} - Removes records from a table.
\end{itemize}

DML commands require knowledge of:
\begin{itemize}
    \item Table column names and data types
    \item Whether columns are primary keys, unique, or allow \texttt{NULL}s
    \item Table and column constraints
\end{itemize}

\section{INSERT Command}
The \texttt{INSERT} command adds new rows to a table. It can be executed in several ways:
\subsection{Basic Syntax}
\begin{verbatim}
INSERT INTO tablename (column1, column2, column3, ...)
VALUES (value1, value2, value3, ...);
\end{verbatim}
The order and number of columns must match between the two lists. Each value must have a compatible data type with its respective column.

\subsection{Insert by Column Positions}
An \texttt{INSERT} without specifying column names relies on column order:
\begin{verbatim}
INSERT INTO tablename VALUES (value1, value2, value3, ...);
\end{verbatim}
This method is generally discouraged, as changes in the table structure may cause it to fail.

\subsection{Insert from Another Table}
The \texttt{INSERT INTO ... SELECT} syntax adds data from another table:
\begin{verbatim}
INSERT INTO tablename
SELECT column1, column2
FROM another_table
WHERE condition;
\end{verbatim}

\subsection{Example}
\begin{verbatim}
CREATE TABLE teams (
    team_id INTEGER NOT NULL PRIMARY KEY,
    team_name VARCHAR(30) NOT NULL,
    city VARCHAR(20) NOT NULL,
    conference VARCHAR(10) NOT NULL
);

INSERT INTO teams (team_id, team_name, city, conference)
VALUES (1, 'Red Wings', 'Detroit', 'Eastern');
\end{verbatim}

\section{UPDATE Command}
The \texttt{UPDATE} command modifies existing records in a table. It can update all rows or a subset of rows based on a condition.
\begin{verbatim}
UPDATE tablename
SET column1 = value1, column2 = value2
WHERE condition;
\end{verbatim}
Using a \texttt{WHERE} clause restricts updates to specific rows.

\subsection{Example with \texttt{WHERE}}
\begin{verbatim}
UPDATE superheroes
SET secret_identity = 'Diana Prince'
WHERE hero_id = 3;
\end{verbatim}

\subsection{Using Comparison Operators}
Comparison operators such as \texttt{=}, \texttt{<>} (not equal), \texttt{<}, \texttt{>}, \texttt{<=}, and \texttt{>=} can be used in the \texttt{WHERE} clause to define conditions. Example:
\begin{verbatim}
UPDATE personnel
SET salary = salary * 1.07
WHERE jobgrade <= 4;
\end{verbatim}

\subsection{Pattern Matching with LIKE}
The \texttt{LIKE} operator uses wildcards for pattern matching:
\begin{verbatim}
UPDATE superheroes
SET gender = 'female'
WHERE hero_name LIKE '%Woman';
\end{verbatim}
Here, \texttt{\%} matches any sequence of characters, while \texttt{\_} matches a single character.

\section{DELETE Command}
The \texttt{DELETE} command removes rows from a table. It does not require column names, as it removes entire rows.
\begin{verbatim}
DELETE FROM tablename
WHERE condition;
\end{verbatim}
A \texttt{WHERE} clause is essential to avoid deleting all rows.

\subsection{Example}
\begin{verbatim}
DELETE FROM superheroes
WHERE hero_id > 3;
\end{verbatim}

\section{Preserving Referential Integrity with Foreign Keys}
When inserting, updating, or deleting rows in tables with foreign key relationships, referential integrity must be maintained:
\begin{itemize}
    \item \textbf{Inserting a row} in a foreign-key table requires the foreign key value to match a primary key in the parent table.
    \item \textbf{Updating a row} in the foreign-key table must ensure the updated foreign key matches an existing primary key in the parent table.
    \item \textbf{Deleting a row} in the parent table may not be allowed if foreign keys in child tables reference that row.
\end{itemize}

\section{Auto-Incrementing IDs}
Primary keys can be set to auto-increment, automatically generating unique values:
\begin{verbatim}
CREATE TABLE counting (
    id INTEGER AUTO_INCREMENT NOT NULL PRIMARY KEY,
    name VARCHAR(10)
);
INSERT INTO counting (name) VALUES ('first');
\end{verbatim}
Auto-increment values continue to increase even if rows are deleted.

\section{Examples of DML Commands with Constraints}
\begin{itemize}
    \item \textbf{Creating Tables with Foreign Keys}
    \begin{verbatim}
    CREATE TABLE authors (
        id INTEGER AUTO_INCREMENT NOT NULL PRIMARY KEY,
        fname VARCHAR(20) NOT NULL,
        lname VARCHAR(25) NOT NULL
    );
    
    CREATE TABLE books (
        id INTEGER AUTO_INCREMENT NOT NULL PRIMARY KEY,
        title VARCHAR(25),
        author_id INTEGER NOT NULL,
        CONSTRAINT author_fk FOREIGN KEY (author_id)
        REFERENCES authors(id)
    );
    \end{verbatim}

    \item \textbf{Inserting Data with Foreign Key Constraint}
    \begin{verbatim}
    INSERT INTO books (title, author_id) VALUES ('Watchers', 1);
    \end{verbatim}
\end{itemize}

\end{document}
