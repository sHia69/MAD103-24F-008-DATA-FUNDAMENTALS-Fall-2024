\documentclass{article}
\usepackage{amsmath}
\usepackage{fancyhdr}

% Header
\pagestyle{fancy}
\fancyhf{}
\fancyhead[L]{MAD 103 - Data Fundamentals}
\fancyhead[R]{Instructor: Sodiq Shofoluwe \thepage}

\begin{document}

\title{SQL Query Basics and Data Retrieval Techniques}
\author{Hia Al Saleh}
\date{October 14th, 2024}
\maketitle
\tableofcontents
\newpage

\section{Introduction to SQL Queries}
SQL (Structured Query Language) is used to interact with relational databases. Basic query commands allow users to retrieve, filter, sort, and analyze data efficiently.

\section{Basic Data Retrieval with \texttt{SELECT}}
\subsection{\texttt{SELECT} Statement}
The \texttt{SELECT} statement is used to retrieve specific columns from a table:
\begin{verbatim}
SELECT column1, column2 FROM tablename;
\end{verbatim}
\texttt{SELECT *} retrieves all columns.

\subsection{Filtering Data with \texttt{WHERE}}
The \texttt{WHERE} clause restricts results to rows that meet specific conditions:
\begin{verbatim}
SELECT column1 FROM tablename WHERE condition;
\end{verbatim}
Common operators include \texttt{=}, \texttt{<>} (not equal), \texttt{>}, \texttt{<}, \texttt{LIKE}, and \texttt{BETWEEN}.

\subsection{Sorting Results with \texttt{ORDER BY}}
\texttt{ORDER BY} arranges the output in ascending (\texttt{ASC}) or descending (\texttt{DESC}) order:
\begin{verbatim}
SELECT column1 FROM tablename ORDER BY column1 DESC;
\end{verbatim}

\section{Aggregate Functions for Data Analysis}
SQL offers several aggregate functions to summarize data:
\begin{itemize}
    \item \texttt{COUNT()} - Counts non-\texttt{NULL} entries.
    \item \texttt{SUM()} - Adds numeric values.
    \item \texttt{AVG()} - Calculates average.
    \item \texttt{MIN()} and \texttt{MAX()} - Find lowest and highest values.
\end{itemize}
Example:
\begin{verbatim}
SELECT COUNT(column) FROM tablename WHERE condition;
\end{verbatim}

\section{Grouping Results with \texttt{GROUP BY} and \texttt{HAVING}}
\subsection{\texttt{GROUP BY} Clause}
\texttt{GROUP BY} organizes rows into groups, allowing aggregate functions on each group:
\begin{verbatim}
SELECT column1, COUNT(*) FROM tablename GROUP BY column1;
\end{verbatim}

\subsection{\texttt{HAVING} Clause}
The \texttt{HAVING} clause filters groups created by \texttt{GROUP BY}:
\begin{verbatim}
SELECT column1, COUNT(*) FROM tablename
GROUP BY column1
HAVING COUNT(*) > 10;
\end{verbatim}

\section{Joins for Combining Tables}
\subsection{Types of Joins}
\begin{itemize}
    \item \textbf{INNER JOIN}: Returns matching rows between tables.
    \item \textbf{LEFT JOIN}: Returns all rows from the left table and matching rows from the right.
    \item \textbf{RIGHT JOIN}: Returns all rows from the right table and matching rows from the left.
    \item \textbf{FULL JOIN}: Returns all rows when there is a match in one of the tables.
\end{itemize}
Example:
\begin{verbatim}
SELECT a.column1, b.column2 FROM table1 AS a
INNER JOIN table2 AS b ON a.id = b.id;
\end{verbatim}

\section{Subqueries}
A subquery is a query nested within another query, often in the \texttt{WHERE} clause:
\begin{verbatim}
SELECT column1 FROM tablename
WHERE column2 = (SELECT MAX(column2) FROM tablename2);
\end{verbatim}

\section{Common SQL Clauses and Operators}
\begin{itemize}
    \item \texttt{DISTINCT}: Removes duplicate rows.
    \item \texttt{IN}: Checks if a value is within a list of values.
    \item \texttt{LIKE} with Wildcards (\texttt{\%}, \texttt{\_}): For pattern matching.
    \item \texttt{BETWEEN}: Filters within a range.
    \item Logical operators: \texttt{AND}, \texttt{OR}, \texttt{NOT}.
\end{itemize}

\end{document}