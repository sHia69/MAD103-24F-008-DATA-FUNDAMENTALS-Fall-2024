\documentclass{article}
\usepackage{amsmath}
\usepackage{fancyhdr}
% Header
\pagestyle{fancy}
\fancyhf{}
\fancyhead[L]{MAD 103 - Data Fundamentals }
\fancyhead[R]{Instructor: Sodiq Shofoluwe  \thepage}

\begin{document}

\title{Notes on Booleans}
\author{Hia Al Saleh}
\date{September 16th, 2024}

\maketitle
\tableofcontents
\newpage 

\section{Booleans}

\begin{itemize}
    \item \textbf{Definition}: Used to store truth values.
    \item \textbf{Type Name}: BOOLEAN.
    \item \textbf{Values}: TRUE, FALSE, UNKNOWN.
    \item \textbf{Null Value}: Equivalent to UNKNOWN.
    \item \textbf{Special Case in Some Languages}:
    \begin{itemize}
        \item MySQL uses a tinyInt value of 0 or 1 for Boolean values.
        \item 1 = true.
        \item 0 = false.
    \end{itemize}
\end{itemize}
\section{Character Data Types}

\begin{itemize}
    \item \textbf{Definition}: Character string (string) types are used to represent text.
    \item \textbf{Strings}:
    \begin{itemize}
        \item Ordered sequences of zero or more characters.
        \item Length can be fixed or varying.
        \item Case sensitive.
        \item In SQL statements, strings are surrounded by single quotes.
        \item The length of a string is an integer between 0 and the specified length.
    \end{itemize}
    \item \textbf{CHAR}:
    \begin{itemize}
        \item Also called CHARACTER.
        \item Requires a specified width (number of characters).
        \item Example: CHAR(50) allows a character string of up to 50 characters.
        \item Excess characters are truncated from the right.
        \item Fixed-length strings are sorted and manipulated faster than variable-length strings.
    \end{itemize}
    \item \textbf{VARCHAR}:
    \begin{itemize}
        \item Varying character type requires a specified width.
        \item Uses only as much storage space as required by the object, up to the set amount.
        \item Example: VARCHAR(50).
    \end{itemize}
    \item \textbf{When to Use String or Numeric}:
    \begin{itemize}
        \item Consider if arithmetic calculations will be performed on the values.
        \item Example: US Postal codes (all digits, fixed length) are best stored as characters.
        \item Example: Telephone numbers (all digits, fixed length) are best stored as characters.
    \end{itemize}
\end{itemize}
\section{Datatypes}

\begin{itemize}
    \item \textbf{Definition}: Indicates the type of data that can be stored in a field.
    \item \textbf{Column Data Type}: Each column has a single data type.
    \item \textbf{Sort Order}:
    \begin{itemize}
        \item The data type affects the column's sort order.
        \item Example: Values 10, 1, 2 are sorted as 1, 2, 10 for integers.
        \item For strings, they are sorted as 1, 10, 2 (lexicographical ordering).
    \end{itemize}
    \item \textbf{Categories of Data}:
    \begin{itemize}
        \item Numeric
        \item Character
        \item Temporal (date and time)
    \end{itemize}
\end{itemize}
\section{Large Objects}

\begin{itemize}
    \item \textbf{CLOB and BLOB}:
    \begin{itemize}
        \item When the character column is larger than the maximum VARCHAR (255), a large-object character type is required.
        \item \textbf{CLOB} (Character Large Object): Used to store large amounts of character data.
        \item \textbf{BLOB} (Binary Large Object): Used to store binary data such as images, sound, and video.
        \item Note: MySQL implements TEXT instead of CLOB.
    \end{itemize}
\end{itemize}
\section{Numeric Data Types}

\begin{itemize}
    \item \textbf{Types}:
    \begin{itemize}
        \item \textbf{Exact}:
        \begin{itemize}
            \item INTEGER: Holds both positive and negative whole numbers. Range: -2,147,483,648 to 2,147,483,647.
            \item SMALLINT: Smaller range of integers. Range: -31,768 to 31,767.
            \item BIGINT: Larger range than INTEGER. Range: -9,223,372,036,854,775,808 to 9,223,372,036,854,775,807.
            \item MEDIUMINT (MySQL): Range: -8,388,608 to 8,388,608.
            \item TINYINT (MySQL and SQL): Range: -128 to 127.
            \item DECIMAL/NUMERIC: Made up of precision (total number of digits) and scale (digits to the right of the decimal point). Example: NUMERIC(5,2) can store 123.89.
        \end{itemize}
        \item \textbf{Approximate}:
        \begin{itemize}
            \item FLOAT: Used for floating point numbers. Example: FLOAT(size, d) where size is the total number of digits and d is the number of digits to the right of the decimal point.
        \end{itemize}
    \end{itemize}
    \item \textbf{Considerations}:
    \begin{itemize}
        \item Exact types ensure the value retrieved is exactly the same as stored.
        \item Approximate types may retrieve a value very close to the original.
        \item Calculations involving only integers are faster than those involving decimal and floating point numbers.
    \end{itemize}
\end{itemize}
\section{Temporal Data Types}

\begin{itemize}
    \item \textbf{DATE}:
    \begin{itemize}
        \item Used to store date values from the Common Era calendar (standard 365-day Gregorian calendar).
        \item Components: Year, Month, Day.
        \item Input formats vary across database systems.
        \item Recommended format: YYYY-MM-DD (recognized by all database systems).
    \end{itemize}
    \item \textbf{TIME}:
    \begin{itemize}
        \item Used to store time values.
        \item Differences exist between input format, storage format, and display format.
        \item Used for recurring clock times and durations.
        \item Based on the 24-hour clock (military time).
        \item Format: hh:mm:ss (colons separate the parts of time).
        \item In MS ACCESS, surround datetime literals with \# (e.g., \#2006-03-17\#).
    \end{itemize}
    \item \textbf{Time Stamp}:
    \begin{itemize}
        \item Consists of both a date and time component.
        \item Used when an event has a specific date and time.
    \end{itemize}
\end{itemize}
\end{document}