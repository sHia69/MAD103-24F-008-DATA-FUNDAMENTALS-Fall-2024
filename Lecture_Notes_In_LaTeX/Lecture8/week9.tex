\documentclass{article}
\usepackage{amsmath}
\usepackage{fancyhdr}

% Header
\pagestyle{fancy}
\fancyhf{}
\fancyhead[L]{MAD 103 - Data Fundamentals}
\fancyhead[R]{Instructor: Sodiq Shofoluwe \thepage}

\begin{document}

\title{Aggregate Functions in SQL}
\author{Hia Al Saleh}
\date{October 28th, 2024}
\maketitle
\tableofcontents
\newpage

\section{Introduction to Aggregate Functions}
Aggregate functions, also called \textbf{set functions}, operate on a group of values to produce a single, summarizing value. Aggregates can be applied to:
\begin{itemize}
    \item All rows in a table
    \item Only specific rows, using a \texttt{WHERE} clause
    \item Rows grouped by a \texttt{GROUP BY} clause
\end{itemize}

Non-aggregate queries process each row independently, while aggregate queries process the table as a whole to construct new rows.

\section{Types of Aggregate Functions}
\subsection{COUNT}
The \texttt{COUNT} function counts the number of records that match a certain criterion.
\begin{itemize}
    \item \texttt{COUNT(*)} - returns the total number of rows in a table, including duplicates and \texttt{NULL}s.
    \item \texttt{COUNT(column\_name)} - returns the number of non-\texttt{NULL} values in the specified column.
\end{itemize}
Example:
\begin{verbatim}
SELECT COUNT(*) AS Num_Books FROM titles;
\end{verbatim}

\subsection{MIN and MAX}
\begin{itemize}
    \item \texttt{MIN(column\_name)} - returns the minimum value in the specified column.
    \item \texttt{MAX(column\_name)} - returns the maximum value.
\end{itemize}
These functions work with character, numeric, and datetime types. Note that \texttt{DISTINCT} has no meaning in \texttt{MIN} and \texttt{MAX}. Examples:
\begin{verbatim}
SELECT MIN(dateOrdered) FROM orders;
SELECT MAX(shippedTo) FROM orders;
\end{verbatim}

\subsection{SUM}
The \texttt{SUM(column\_name)} function calculates the sum of numeric values in the specified column.
\begin{itemize}
    \item It works only with numeric data types.
    \item \texttt{NULL} values are ignored.
    \item If there are no rows to sum, it returns \texttt{NULL} (not zero).
\end{itemize}
Example:
\begin{verbatim}
SELECT SUM(price) FROM carInventory WHERE qtyInStock = 1;
\end{verbatim}

\subsection{AVG}
The \texttt{AVG(column\_name)} function calculates the average of numeric values in the specified column.
\begin{itemize}
    \item Works only on numeric data types.
    \item If no rows match, it returns \texttt{NULL}.
\end{itemize}
Example:
\begin{verbatim}
SELECT AVG(price) FROM carInventory WHERE make = 'Mazda';
\end{verbatim}

\section{Other Considerations with Aggregate Functions}
\begin{itemize}
    \item \textbf{Ignoring \texttt{NULL}s}: All aggregate functions except \texttt{COUNT(*)} ignore \texttt{NULL} values.
    \item \textbf{Aliases}: Use aliases to provide meaningful names for aggregate results in the \texttt{SELECT} clause.
\end{itemize}

\section{Restrictions on Aggregate Functions}
\begin{itemize}
    \item Aggregate functions cannot be used in the \texttt{WHERE} clause. Attempting to use one will result in an error.
    \item You cannot mix non-aggregate and aggregate columns directly in the \texttt{SELECT} clause without using a \texttt{GROUP BY}.
    \item Nested aggregate functions are not allowed. For example:
    \begin{verbatim}
    SELECT SUM(AVG(sales)) FROM titles;
    \end{verbatim}
    \item Subqueries cannot be used within aggregate expressions.
\end{itemize}

\section{Using \texttt{GROUP BY} with Aggregate Functions}
The \texttt{GROUP BY} clause is used to divide a table into groups and apply aggregate functions to each group. The database processes rows by:
\begin{itemize}
    \item Gathering all information from the \texttt{FROM} clause
    \item Filtering by the \texttt{WHERE} clause, if present
    \item Aggregating the rows into groups
\end{itemize}
Only grouped rows can be used in the \texttt{SELECT} clause. Examples:
\begin{verbatim}
SELECT make, COUNT(*) AS numberOf
FROM carInventory
GROUP BY make;

SELECT make, modelYear, COUNT(*) AS numberOf
FROM carInventory
GROUP BY make, modelYear;
\end{verbatim}

\section{DISTINCT with Aggregate Functions}
\texttt{DISTINCT} can be used to eliminate duplicate values within aggregate functions, but it’s not meaningful with \texttt{MIN} and \texttt{MAX}. Example:
\begin{verbatim}
SELECT COUNT(DISTINCT make) FROM carInventory;
\end{verbatim}

\section{\texttt{HAVING} Clause}
The \texttt{HAVING} clause filters results after aggregation, working similarly to the \texttt{WHERE} clause but applying to aggregated data. It is optional and used only with \texttt{GROUP BY}. Rows removed by the \texttt{WHERE} clause are not available to the \texttt{HAVING} clause.
\begin{verbatim}
SELECT make, SUM(price) AS totalPrice
FROM carInventory
GROUP BY make
HAVING SUM(price) > 2000000;
\end{verbatim}

\section{Summary of Aggregate Function Usage}
Aggregate functions enable summarizing data in SQL tables:
\begin{itemize}
    \item \texttt{COUNT}, \texttt{SUM}, and \texttt{AVG} for numeric summaries.
    \item \texttt{MIN} and \texttt{MAX} for finding boundary values.
\end{itemize}
\texttt{GROUP BY} and \texttt{HAVING} clauses allow further control and filtering of results at the group level, making aggregate functions versatile tools in data summarization.

\end{document}