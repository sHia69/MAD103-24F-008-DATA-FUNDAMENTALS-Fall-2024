\documentclass{article}
\usepackage{amsmath}
\usepackage{listings}
\usepackage{geometry}
\usepackage[utf8]{inputenc}
\geometry{margin=1in}

\title{JSON Self-Study Questions}
\author{Hia Al Saleh}
\date{September 16th, 2024}

\begin{document}

\maketitle

\section*{Question 1: JSON Datatypes}

\textbf{Question:} The following are all valid JSON datatypes EXCEPT for:

\begin{itemize}
  \item Array
  \item Integer
  \item String
  \item Object
\end{itemize}

\textbf{Answer:} The correct answer is \textbf{Integer}. JSON does not have a specific "Integer" type. Instead, it has the \textbf{Number} datatype, which includes both integers and floating-point numbers.

\section*{Question 2: JSON Stands For}

\textbf{Question:} JSON stands for:
\begin{itemize}
  \item JavaScript Object Notation
  \item JavaScript Schema Object Notation
  \item JavaScript Scripted Object Notation
  \item JavaScript Schema Oriented Notation
\end{itemize}

\textbf{Answer:} The correct answer is \textbf{JavaScript Object Notation}.

\section*{Question 3: Symbol Separating Name-Value Pairs}

\textbf{Question:} This symbol separates multiple name-value pairs in JSON:
\begin{itemize}
  \item :
  \item \{ \}
  \item [ ]
  \item ,
\end{itemize}

\textbf{Answer:} The correct answer is \textbf{,} (comma). Commas are used to separate name-value pairs within objects and values in arrays.

\section*{Question 4: Structure of JSON}

\textbf{Question:} JSON is made up of:
\begin{itemize}
  \item name:value
  \item elements:value
  \item string:value
  \item value:name
\end{itemize}

\textbf{Answer:} The correct answer is \textbf{name:value}. JSON consists of key-value pairs where the key is a string and the value can be any valid JSON datatype.

\section*{Question 5: Validity of JSON}

\textbf{Question:} The following is valid JSON:

\begin{lstlisting}
{
  "name": "Doug"
}
\end{lstlisting}

\textbf{True or False?}

\textbf{Answer:} The correct answer is \textbf{True}. The provided JSON is valid as it contains a properly formatted key-value pair.

\section*{Question 6: Valid JSON Examples}

\textbf{Question:} This is valid JSON:
\begin{itemize}
  \item \{ overAge: TRUE \}
  \item \{studentName: "Frank"\}
  \item \{ "imageLocation" : "C:\\images" \}
  \item \{ "numberOfStudents": 2 \}
\end{itemize}

\textbf{Answer:} The correct answer is \textbf{\{ "numberOfStudents": 2 \}}. The other options are invalid because:
\begin{itemize}
  \item Keys must be enclosed in double quotes.
  \item The boolean value \texttt{TRUE} should be in lowercase (\texttt{true}).
  \item Backslashes must be escaped.
\end{itemize}

\section*{Question 7: Null Datatype}

\textbf{Question:} The null datatype is used to represent a value that cannot be defined, such as an empty string or the value 0.

\textbf{True or False?}

\textbf{Answer:} The correct answer is \textbf{False}. The null datatype in JSON explicitly represents the absence of a value, which is different from an empty string ("") or 0, which are valid values.

\section*{Question 8: Validity of JSON}

\textbf{Question:} The following is valid JSON:

\begin{lstlisting}
{
  "numberOfPlayers": 6
}
\end{lstlisting}

\textbf{True or False?}

\textbf{Answer:} The correct answer is \textbf{True}. The provided JSON is valid, with a correct key-value pair and syntax.

\section*{Question 9: Escaping Characters in JSON}

\textbf{Question:} The following is valid JSON:

\begin{lstlisting}
{
  "quote" : "He said \"Come out to the coast, we’ll get together, have a few laughs\""
}
\end{lstlisting}

\textbf{True or False?}

\textbf{Answer:} The correct answer is \textbf{False}. The inner double quotes need to be escaped using backslashes (\texttt{\textbackslash}). The corrected JSON would look like this:

\begin{lstlisting}
{
  "quote": "He said \"Come out to the coast, we’ll get together, have a few laughs\""
}
\end{lstlisting}

\end{document}