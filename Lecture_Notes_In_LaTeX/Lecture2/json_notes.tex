\documentclass{article}
\usepackage{amsmath}
\usepackage{geometry}
\usepackage{fancyhdr}
\geometry{a4paper, margin=1in}

\pagestyle{fancy}
\fancyhf{}
\fancyhead[L]{MAD 103 - Data Fundamentals - Fall 2024}
\fancyhead[R]{Instructor: Sodiq Shofoluwe  \thepage}

\title{Notes on JSON}
\author{Hia Al Saleh}
\date{September 9th, 2024}

\begin{document}

\maketitle
\tableofcontents
\newpage 


\section{Final Exercise}

\begin{itemize}
    \item \textbf{Objective}: Create a valid JSON document using the given information.
    \item \textbf{Guidance}:
    \begin{itemize}
        \item Structure the document with appropriate key-value pairs.
        \item Use meaningful names for the keys.
        \item Verify the validity of the JSON after completion.
    \end{itemize}
\end{itemize}

\section{What is JSON}

\begin{itemize}
    \item \textbf{Definition}: JSON stands for \textit{JavaScript Object Notation}.
    \item \textbf{Purpose}: 
    \begin{itemize}
        \item Text format used for exchanging data between platforms.
        \item It is a subset of JavaScript but works independently of it.
    \end{itemize}
    \item \textbf{Characteristics}:
    \begin{itemize}
        \item Portable, works across platforms and systems.
        \item Based on the JavaScript object literal syntax.
        \item Simple and easy to read.
        \item Files use the \texttt{.json} extension.
        \item Shorter than XML, as it does not require closing tags, leading to smaller file sizes and faster transfer.
    \end{itemize}
\end{itemize}

\section{How is JSON Formatted}

\begin{itemize}
    \item \textbf{Syntax}:
    \begin{itemize}
        \item JSON is marked like a JavaScript object with \texttt{\{\}} for objects and \texttt{[]} for arrays.
        \item Key-value pairs are separated by colons (\texttt{:}), and multiple pairs are separated by commas (\texttt{,}).
    \end{itemize}
    \item \textbf{Types of JSON Values}:
    \begin{itemize}
        \item JSON values can include: Objects, Strings, Numbers, Booleans, Null, and Arrays.
    \end{itemize}
    \item \textbf{Rules}:
    \begin{itemize}
        \item Keys (names) must always be in double quotes.
        \item Values only require quotes depending on the datatype (e.g., strings).
        \item Escape characters like backslashes and quotes must be used appropriately.
    \end{itemize}
    \item \textbf{Data Types}:
    \begin{itemize}
        \item \textbf{Numbers}: Can be integers, decimals, negative numbers, or exponents, and do not require quotes.
        \item \textbf{Boolean}: Values must be lowercase (\texttt{true} or \texttt{false}).
        \item \textbf{Null}: Indicates no value (not the same as undefined in JavaScript).
    \end{itemize}
    \item \textbf{Arrays}:
    \begin{itemize}
        \item Always enclosed in square brackets (\texttt{[]}), containing valid JSON data types.
    \end{itemize}
    \item \textbf{Nested Objects}:
    \begin{itemize}
        \item JSON allows objects to be nested within other objects, using a tree-like structure.
    \end{itemize}
    \item \textbf{Portability Considerations}:
    \begin{itemize}
        \item Avoid special characters or spaces in key names for compatibility across platforms.
    \end{itemize}
\end{itemize}

\end{document}