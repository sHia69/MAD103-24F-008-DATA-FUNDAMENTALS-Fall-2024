\documentclass{article}
\usepackage{amsmath}
\usepackage{fancyhdr}

% Header
\pagestyle{fancy}
\fancyhf{}
\fancyhead[L]{MAD 103 - Data Fundamentals}
\fancyhead[R]{Instructor: Sodiq Shofoluwe \thepage}

\begin{document}

\title{Database Design and Normalization Notes}
\author{Hia Al Saleh}
\date{November 25th, 2024}
\maketitle
\tableofcontents
\newpage

\section{Introduction}
\subsection{Overview of the Relational Model}
The relational model forms the foundation of modern database systems. It organizes data into relations, commonly referred to as tables, each representing a specific entity type. 

\paragraph{Key Characteristics of a Table:}
\begin{itemize}
    \item A table contains rows and columns, where:
    \begin{itemize}
        \item \textbf{Rows:} Represent individual records or facts about an entity.
        \item \textbf{Columns:} Represent attributes or properties of the entity.
    \end{itemize}
    \item Tables must have a unique name within the database.
    \item Data items are stored at the intersection of rows and columns, ensuring a structured organization.
\end{itemize}

\paragraph{Example:} Consider a table called \texttt{Student}, which stores information about students:
\begin{itemize}
    \item Columns: \texttt{Student\_ID}, \texttt{Name}, \texttt{Date\_of\_Birth}.
    \item Rows: Each row corresponds to a specific student and includes values for the attributes.
\end{itemize}

\subsection{Entities and Attributes}
An \textbf{entity type} refers to a class of real-world objects, such as patients, movies, or invoices. Each entity type has attributes that define its properties. For example:
\begin{itemize}
    \item \textbf{Entity Type:} \texttt{Movie}
    \item \textbf{Attributes:} \texttt{Title}, \texttt{Release Year}, \texttt{Director}.
\end{itemize}

\subsection{Rows and Columns in Detail}
\paragraph{Columns (Attributes):}
\begin{itemize}
    \item Each column has a name and a \textbf{domain}, which defines the allowable data types and values.
    \item Values in columns must be single-valued (atomic).
    \item The order of columns in a table is unimportant.
\end{itemize}

\paragraph{Rows (Records):}
\begin{itemize}
    \item Each row describes a specific instance of the entity.
    \item Rows must be unique and are identified using a \textbf{primary key}.
    \item The order of rows is unimportant.
\end{itemize}

\section{Relationships}
\subsection{Definition}
A \textbf{relationship} is an association between tables established through common columns. Relationships ensure that data in different tables is interconnected, forming the backbone of a relational database.

\subsection{Types of Relationships}
\paragraph{One-to-One:}
\begin{itemize}
    \item Each row in Table A is associated with at most one row in Table B, and vice versa.
    \item Example: A table of \texttt{Employees} linked to a table of \texttt{Company Cars}.
    \item Implementation: The primary key of one table also acts as the foreign key in the related table.
\end{itemize}

\paragraph{One-to-Many:}
\begin{itemize}
    \item A single row in Table A can be linked to multiple rows in Table B.
    \item Example: A table of \texttt{Authors} linked to a table of \texttt{Books}.
    \item Implementation: The primary key of Table A is included as a foreign key in Table B.
\end{itemize}

\paragraph{Many-to-Many:}
\begin{itemize}
    \item Rows in Table A can relate to multiple rows in Table B, and vice versa.
    \item Example: A table of \texttt{Students} linked to a table of \texttt{Courses}.
    \item Implementation: A junction table is created to store the primary keys of both tables.
\end{itemize}

\section{Normalization}
Normalization is the process of organizing data in a database to reduce redundancy, avoid update anomalies, and maintain consistency.

\subsection{First Normal Form (1NF)}
\paragraph{Definition:}
A table is in 1NF if it has no repeating groups and all its columns contain atomic (indivisible) values.

\paragraph{Steps to Achieve 1NF:}
\begin{itemize}
    \item Eliminate repeating groups by storing each value in its own row.
    \item Ensure each column contains a single value.
\end{itemize}

\paragraph{Example:}
\textbf{Unnormalized Table:}
\begin{center}
\begin{tabular}{|l|l|l|}
\hline
Student ID & Student Name & Courses Enrolled \\ \hline
1 & John & Math, Science \\ \hline
\end{tabular}
\end{center}

\textbf{1NF Table:}
\begin{tabular}{|l|l|l|}
\hline
Student ID & Student Name & Course \\ \hline
1 & John & Math \\ \hline
1 & John & Science \\ \hline
\end{tabular}

\subsection{Second Normal Form (2NF)}
\paragraph{Definition:}
A table is in 2NF if:
\begin{itemize}
    \item It is in 1NF.
    \item Every non-key attribute depends on the entire primary key.
\end{itemize}

\paragraph{Steps to Achieve 2NF:}
\begin{itemize}
    \item Identify attributes that depend only on a part of a composite key.
    \item Create separate tables for these attributes.
\end{itemize}

\paragraph{Example:}
Splitting a table of \texttt{Students} and \texttt{Courses} into separate \texttt{Student} and \texttt{Course} tables.

\subsection{Third Normal Form (3NF)}
\paragraph{Definition:}
A table is in 3NF if:
\begin{itemize}
    \item It is in 2NF.
    \item No attribute depends on a non-key attribute (transitive dependency).
\end{itemize}

\paragraph{Steps to Achieve 3NF:}
\begin{itemize}
    \item Identify attributes dependent on non-key attributes.
    \item Move these attributes into a new table.
\end{itemize}

\paragraph{Example:}
Splitting a table of \texttt{Students}, \texttt{Courses}, and \texttt{Coordinators} into separate tables.

\section{Conclusion}
The relational model and normalization techniques are essential for designing efficient databases. By ensuring data integrity and minimizing redundancy, these principles lay the groundwork for scalable and maintainable database systems.

\end{document}