\documentclass{article}
\usepackage{amsmath}
\usepackage{fancyhdr}

% Header
\pagestyle{fancy}
\fancyhf{}
\fancyhead[L]{MAD 103 - Data Fundamentals - Fall 2024}
\fancyhead[R]{Instructor: Sodiq Shofoluwe  \thepage}


\title{XML and Data Fundamentals Notes}
\author{Hia Al Saleh}
\date{September 5th,2024}

\begin{document}

\maketitle
\tableofcontents
\newpage 

\section{Introduction to XML}
\begin{itemize}
    \item XML stands for \textbf{Extensible Markup Language}, a technology used for data description and structuring.
    \item It is not a language itself but a standard for creating languages that meet XML criteria.
    \item XML files are text files with the \texttt{.xml} extension and are maintained by the World Wide Web Consortium (W3C).
\end{itemize}

\section{XML Document Structure Rules}
\begin{enumerate}
    \item Tags are case-sensitive.
    \item Opening tags must have corresponding closing tags.
    \item Tags must be properly nested.
    \item Attribute values must be quoted.
    \item The root element must be unique and required.
\end{enumerate}

\section{Basic XML Example}
\begin{verbatim}
<books>
    <book>
        <title>Cold Days</title>
        <author>Jim Butcher</author>
    </book>
</books>
\end{verbatim}

\section{XML Declaration}
\begin{itemize}
    \item XML declarations include the version and character encoding, e.g., \texttt{<?xml version="1.0" encoding="UTF-8"?>}.
    \item The \texttt{standalone} attribute indicates if the document relies on external files.
\end{itemize}

\section{Character Data and Entities}
\begin{itemize}
    \item Predefined entities in XML: \texttt{\&lt;} (represents \texttt{<}), \texttt{\&gt;} (represents \texttt{>}), \texttt{\&amp;} (represents \texttt{\&}), \texttt{\&apos;} (represents \texttt{'}) and \texttt{\&quot;} (represents \texttt{"}).
    \item CDATA sections are used to include large blocks of unparsed text: \texttt{<![CDATA[...]]>}.
\end{itemize}

\section{Data Fundamentals}
\subsection{Definition of Data}
\begin{itemize}
    \item Factual information used for reasoning or calculation.
    \item Data can be transmitted or processed digitally.
\end{itemize}

\subsection{Popular Data Formats}
\begin{itemize}
    \item XML (Extensible Markup Language).
    \item JSON (JavaScript Object Notation).
\end{itemize}

\section{Course Information}
\subsection{Important Dates}
\begin{itemize}
    \item Start Date: September 3, 2024
    \item End Date: December 20, 2024
    \item Add/Drop Deadline: September 16, 2024
    \item Thanksgiving Break: October 14, 2024
    \item Exam Week: December 9-13, 2024
\end{itemize}

\subsection{Grading Breakdown}
\begin{itemize}
    \item Assignments: 30\% total, with each worth 2.5\%.
    \item Quizzes: 10\% total, with each quiz worth 1\%.
\end{itemize}

\section{Email Etiquette}
\begin{itemize}
    \item Use professional language in all communication.
    \item Include your full name, student number, and course section in all emails.
    \item Always use your college email account for correspondence.
\end{itemize}

\end{document}